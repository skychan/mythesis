% !Mode:: "TeX:UTF-8"
% !TEX root = ..\Literature_Translation.tex
\kchapter{建立解决方法}
本章我们将建立该问题的解决方法。首先,我们先将引入作业划分概念,并结合批量作业以减少换线时间、提高装配车间效率。而后,我们建立两个最优化特点,并提出了一个整数规划(MIP)公式以导出最优解。最后,我们提出了三个启发式方法以得到近似最优解。
\ksection{作业划分和批量作业}
为了改善现行的调度方法,我们引入了多批次发货(Bukchin,2002)概念,并打算用作业划分策略将作业划分成几个子作业。将作业划分策略结合入批量处理后,$c$项子作业便可分开同时由$c$个工人来处理。这样一来,可以避免相互干涉,减少等待时间。由于这些子作业都在批量中心,并且是连续离开机器,这$c$位工人必须同时停止作业,否则一些工人可能会处于闲置状态,而另一些工人却还在忙。
可以运用模块化设计和标准化以确保这$c$位工人同时停止作业,这样一来所工作拥有同样的处理时间,$c$位工人可以的工作强度近似一致。
如有必要,这4位工人可以互相帮工,这样一来,这些子作业可以同时完成。
因此,处理可以变得更为流畅,并且在工段1不会产生瓶颈。如\reff{fig:bwf}所示,接下来整个系统可以看作一个两阶段流水车间,工段1有4工人操作的批处理机器,工段2有2工人操作的离散机器。

承前所述,我们在工段1将作业划分成$c$项子作业,必须由$c$位工人同时操作,这样他们可以看作一个容量为$c$项作业的批量处理机器(Liu 和Yu,2000)。批$i$的处理时间为在此批中的最长作业处理时间,即$p_1(B_i)=\max p_{1j}=p_1$。
此外,每当一个新的批量产生,都需要考虑批量换线时间$s_b$,同样,开始首项作业或切换不同产品簇的作业时,需要考虑产品簇换线时间$s_{f1}$。
需要注意的是,批量中的最后一个作业或者同产品簇的下批首个作业不需要考虑产品簇换线时间。如此一来,批次$i$的完工时间$C_{1,[i]}$等于批量启动时间加上批量换线时间加上同批的产品簇换线和作业处理时间,即:
\begin{gather}
C_{1,[0]}=0 \label{} \\
C_{1,[i]}=C_{1,[i-1]} + s_b + s_{f1}\sum_{r=1}^c k_{1,[i,r]} + p_1 \\
(i=1,...,b, \text{如果}f_{1,[i,r]}=f_{1,[i,r-1]}\ k_{1,[i,r]}=0, \text{否则}k_{1,[i,r]}=1 )\notag
\end{gather}

作业在工段2用离散机器处理,作业按批进入但是一个接一个离开。开始首项工作或者切换产品簇的时候需要考虑产品簇换线时间$s_{j2}$。$J_{[j]}$的完工时间$C_{2,[j-1]}$等于$C_{1,[i]}$或者$C_{2,[j-1]}$加上$J_{[j]}$的产品簇换线时间和作业处理时间,即:
\begin{gather}
C_{2,[0]}=0\\
C_{2,[j]}=\max\{C_{1,[i]},C_{2,[j-1]}\} + s_{f2}\times k_{2,[j]} + p_{2,[j]}\\
(i=1,...,b,j=1,...,n \text{如果}f_{2,[j]}=f_{2,[j-1]}\ k_{2,[j]}=0, \text{否则}k_{2,[j]}=1)\notag 
\end{gather}

如此一来,
\begin{gather}
C_{\max}=C_{2,[n]}\\
\sum_{j=1}^n T_j = \sum_{j=1}^n \max\{C_{2,[j]}-d_{[j]},0\}\\
Z = \alpha C_{\max} + \beta\sum_{j=1}^n C_{2,[j]} + \gamma\sum_{j=1}^n T_j
\end{gather}

为了提高一次完成率,我们只接受滞后和小于$20\ h$每周的调度,因为滞后作业可以适当的周末里加班解决。否则,生产管理员需要和销售人员协商更改滞后作业的完工时间并重新调度。
\ksection{发展属性}
