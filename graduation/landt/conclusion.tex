% !Mode:: "TeX:UTF-8"
% !TEX root = ..\Literature_Translation.tex
\chapter{总结}
先总结文献综述的内容,然后指出文献综述中所做的事情对正式撰写论文有什么帮助和支持

调度问题本身具有高复杂性、多约束性、多目标性,在多品种小批量生产方式下,由于产品品种多、批量小、零件多等特点,增加了调度的复杂程度。多品种的装配调度问题是一种常见的车间调度问题,对于汽车零件,又有其独有的工艺特点。
本文针对多品种装配车间调度问题分析了部分国内外相关的研究。国内学者在调度的算法改造中有很多创新,在其研究的课题中体现出优良的性能,实用价值很大,对本课题的算法研究启发颇多,国外学者在研究调度算法时,一般会将多种方法结合使用,而且新方法较多,发展空间大,对本课题的研究很有借鉴意义。

随后,本文简要说明了调度的相关理论,确立了确定型和随机型的模型框架和概念。有关调度问题的算法众多,而各方法的适用对象亦不尽相同。对于多品种的装配调度,需要考虑的目标较多,许多目标间可能存在矛盾关系,需要有合理的解决办法。另一方面,合理的约束条件可以使问题更符合实际,但同时也增加了难度。本文之后选取了3个与本课题相关度较高的方法,FBFS,VNS,PSO,涉及目标函数的优化及解的搜索。FBFS 算法的特点是将作业按簇分成批次,可以有效减少作业簇准备时间,适合多品种的生产调度,然而该方法只按优先级排序作业,按批次的特点安排生产,可能会导致部分作业滞后完成而部分作业却提前完成。这个方面可以通过例如滚动时域的方法将其改善。VNS 可通过扰动来使之跳出局部最优,虽然这个方法在解决多目标优化问题的时候效果较好,然而随着目标的复杂程度增加,搜索空间大小将呈现指数增长,求解困难增加。可以通过一些方法,如分支定界或有限差异搜寻等提高搜索质量。同时,也可通过神经网络的方法来得到各目标间的权衡系数,以提高目标函数质量。POS 技术只需初级和简单数学操作即可完成,通过粒子的速度和位置两个部分,形象了搜索过程,能较快达到近似最优解,所需时间和成本都较低,能够较好处理目标很多的柔性流水车间调度,然而其能处理的问题规模较小。对于能拆分成小型问题的中大型问题,此方法就很有优势。此外要关注起收敛速度。