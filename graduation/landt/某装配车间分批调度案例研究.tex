% !Mode:: "TeX:UTF-8" 
\stepcounter{app}
\begin{Abstract}
\chapter*{某装配车间分批调度案例研究}\addcontentsline{toc}{section}{某装配车间分批调度案例研究}
\begin{center}
\vspace{2mm}
{
 {\xiaosi Rock Lin, Ching-Jong Liao}

 {\xiaowu Department of Industrial Management, National Taiwan University of Science and Technology, Taipei 106, Taiwan}
}
\end{center}
{\wuhao \songti 
\noindent \textbf{摘要:}本文探讨了在某机械工厂装配车间内的生产调度问题。该装配工艺包含两个阶段:在第一阶段,所需零件在一批同速机上同时装配,并且这些同速机的准备时间也相同;装配完成的部件进入第二阶段进,并在不同的异速机上进行系统集成组装。同速机和异速机在切换生产产品簇的时候都需要考虑换线时间。本文建立了一个混合整数规划(MIP)模型以求解小型问题,并提出了用于求解中、大型问题的三个启发式方法。经过计算检验,相比较其余两个方法,其中一个利用滚动时域调度策略的整批产品簇排序启发式组合的方法(RFBFS),在解决问题方面有较高质量。实践表明,RFBFS方法确实显著优于现行方法。

\keywords{混合整数规划、作业划分、批量生产、产品簇调度}
}
\end{Abstract}

\kchapter{引言}
在机械制造工厂中,经常会遇到批量调度问题,我们将在本文具体研究工厂中的两阶段组装车间的调度。这个问题与Yokoyama 等的论文中所描述的两阶段组装车间调度问题类似:第1阶段,生料组装成产品部件;第2阶段,这些部件装配成产品。然而,由于技术与工艺十分复杂,生料的组装不仅困难,而且成本高,这便是通常情况下,企业将生料组装外包(或直接采购组装件)的原因。因此,我们为这个装配车间构建成一个两阶段流水车间模型,在第1阶段进行模块化装配,在第2阶段进行系统集成。

随着20年的技术不断革新,需求的多样性也水涨船高,产品便不断向着多功能、高性能的要求发展,这使得机械化生产变得日益复杂。如此背景下,工厂的车间管理不得不处理好下列调度问题:
\begin{compactenum}[(1)]
\item 如何缩短生产周期?
\item 如何减少在制品(WIP)?
\item 如何实现及时交货?
\item 如何处理紧急插单?
\end{compactenum}

本文的研究对象为台湾的一家机械厂——峰明机械有限公司,该工厂能按订单制造10余种塑料袋生产机。
% !Mode:: "TeX:UTF-8"
% !TEX root = ..\Literature_Translation.tex
\kchapter{现行调度方法}
下面将引入相关符号,并对案例中的工厂所运用的现行调度方法进行描述。
\ksection{符合说明与相关假设}
为了方便问题描述,符号说明如下:%\\[5pt]

\begin{supertabular}{ll}
$j$ & 作业标记 \\
$i$ & 批次标记\\
$r$ & 批次中的位置标记\\
$b$ & 工段中的批数,$b\geqslant2$\\
$c$ & 工段1的批量\\
$J$ & 需要调度的作业集合,$J=\{1,2,...,n\}$\\
$p_{aj}$ & $J_j$在工段$a$的单件处理时间,$p_{aj}=p_a$\\
$p_{1j}$ & $J_j$在工段1的处理时间,$p_{1j}=p_1$\\
$p_{2j}$ & $J_j$在工段1的处理时间\\
$s_a$ & 在工段1的作业换线时间\\
$s_b$ & 在工段2的批次换线时间\\
$s_{f1}$ & 在工段1的产品簇换线时间\\
$s_{f2}$ & 在工段2的产品簇换线时间\\
$f_j$ & $J_j$的产品簇,$f_j=1,2,3,4$\\
$f_{1,[i,r]}$ & 在工段1中安排在第$r$批次作业$[i,r]$的产品簇 \\
$f_{2,[j]}$ & 在工段2中安排在第$j$批次$J_{[j]}$的产品簇 \\
$k_{1,[i,r]}$ & \begin{numcases}{=}
{\liuhao 0 }&{\liuhao 在工段1,如果作业$[i,r]$和它的前继作业同属一个产品簇}\notag\\
{\liuhao 1 }&{\liuhao 其他情况} \notag
\end{numcases}\\[5pt]
$k_{2,[j]}$ & \begin{numcases}{=}
{\liuhao 0 }&{\liuhao 在工段2,如果$J_[j]$和它的前继作业同属一个产品簇}\notag\\
{\liuhao 1 }&{\liuhao 其他情况} \notag
\end{numcases}\\
$S$ & 已调度的作业集合\\
$S_U$ & 未调度作业集合\\
$B$ & 已调度批集合\\
$B_i$ & 工段1的第$i$批,$B_i\subset B$\\
$p_1(B_i)$ & 工段1的$B_i$处理时间\\
$D_j$ & $J_j$的交货期\\
$d_j$ & $J_j$的工期\\
$d_{[j]}$ & $J_{[j]}$在工段2的工期\\
$C_{1,[i]}$ & 批$i$在工段1的完工时间\\
$C_{2,[j]}$ & $J_{[j]}$在工段2的完工时间\\
$T_j$ & $J_{[j]}$的滞后时间\\[10pt]
%$$ & \\
\end{supertabular}
%\end{center}

下面是该问题的相关假设:
\begin{itemize}
\itemsep=0pt\parskip=0pt
\item 所有作业在0时刻皆未被占用;
\item 作业总量可被批量大小$c$整除,即$n=b\times c$;
\item 批次中的作业需在工段1处理完该批后方可开始工段2的处理;
\item 工段1中处理单件需要考虑作业换线时间$s_a$;
\item 工段1上的批量处理需要考虑批量换线时间$s_b$;
\item 新产品簇在工段1上加工前需要考虑产品簇换线时间$s_{f1}$;
\item 新产品簇在工段2上加工前需要考虑产品簇换线时间$s_{f2}$;
\item 事先已获得固定的处理时间、换线时间和交付期。
\end{itemize}
\ksection{生产制造过程}
为了简化流程和扩大生产能力,所有零件、组装子件全部外包给资深的合作伙伴或零售商。因此,该生产系统可以被构建成一个两阶段装配流水车间:
\begin{inparaenum}[(1)]
\item 一个组装工段,用来将零部件、模块等组装成框架;
\item 一个集成工段,应用了机械电子工程(M\&E)的集成手段。
\end{inparaenum}
应用模块化设计和标准化以允许所有产品共享同条生产线,并且有着相同的生产时间。如此一来,所有产品都可由11个子系统构成,如\reft{tab:11+4}所示。
\begin{table}[h]
  \centering\xiaowu
  \caption{11个子系统和4个作业块(A,B,C,D)}
    \begin{tabular}{llllll}
    \toprule
    \multicolumn{2}{c}{前部} & \multicolumn{2}{c}{中部} & \multicolumn{2}{c}{后部} \\
    \midrule
    A-1   & 推辊机   & B-1   & 水平电子码(EPC)机 & D-1   & 袋酱机 \\
    A-2   & 张力锟机  & B-2   & 垂直电子码(EPC)机 & D-2   & 冲孔机 \\
    A-3   & 压袋机   & B-3   & 封刀机   & D-3   & 输送机 \\
    A-4   & 送料机   & C-1   & 摇摆系统  &       &  \\
    \bottomrule
    \end{tabular}%
  \label{tab:11+4}%
\end{table}%
需要考虑各子系统的负荷和操作进程,以平衡产线和缩短完工时间(Nearchou,2011)。所有的操作被分解成4个操作块(A,B,C,D),并且分别指派给4位一组中的熟练工。这4位工人将框架和底架移至工段1的同一个区域以开始作业的操作,同时他们需要准备好操作的所需部件和工具。而后,他们便开始在各自的操作块中进行组装。
由于空间限制,组装操作很容易产生人员间的干涉,并会由此产生潜在的等待时间,这常常会导致生产瓶颈。
每当作业在工段1的组装操作结束,便会转移到工段2,由2位该组的熟练工继续作业。他们运用机电集成技术,包括伺服或计算机系统安装、参数设置、调试等。之后,完成的作业被移至成品区。

\ksection{换线时间和产品簇}
大多数有关调度的研究将换线时间看做是可忽略或者将其算作处理时间的一部分(Logendran,2005;Allahverdi,1999),但是在此案例中,换线时间是很重要的,不得不考虑。此处,换线时间可以归为两类:
\begin{inparaenum}
\item 作业换线,作为工作准备,将所有要处理作业的零部件置好;
\item 产品簇换线,用于获得工具、调整工装夹具、检查不同产品簇的需要加工零部件(Eren,2007)。
\end{inparaenum}
工段1在操作开始前,需要一个作业换线时间$s_a$,而当开始第1个作业或切换产品簇的时候,在两个工段都需要考虑产品簇换线时间$s_{j1}$和$s_{j2}$。简化起见,我们假设这些都是常数,即$s_a=5.2\ h$,$s_{f1}=3.2\ h$,$s_{f2}=1.6\ h$。如\reft{tab:2yearproduction}所示,现有超过10个模块在案例工厂加工,并且这些作业可以根据模块属性群组为4个产品簇($f_1\sim f_4$)。需要注意作业处理时间不包括换线时间$s_a$,因为它需要和批次换线时间换线时间$s_b$作比较。
\begin{table}[htbp]
  \centering\xiaowu
  \caption{近两年作业的模块和产品簇的产量}
    \begin{tabular}{lrcccc}
    \toprule
    \multicolumn{2}{c}{模块} & \multicolumn{2}{c}{数量} & \multicolumn{2}{c}{处理时间(h)} \\
    \midrule
    \multicolumn{1}{l}{编号} & \multicolumn{1}{c}{产品簇 } & 2007  & 2008  & $p_1/1$位工人 & $p_2/2$位工人 \\
    \midrule
    900BR & \multicolumn{1}{c}{$f_1$} & 26    & 15    & 24    & 16 \\
    1000DT & \multicolumn{1}{c}{$f_1$} & 30    & 46    & 24    & 16 \\
    800ST2 & \multicolumn{1}{c}{$f_2$} & 44    & 51    & 24    & 12 \\
    1100ST3 & \multicolumn{1}{c}{$f_2$} & 18    & 11    & 24    & 12 \\
    1000TT & \multicolumn{1}{c}{$f_3$} & 43    & 30    & 24    & 8 \\
    800TT & \multicolumn{1}{c}{$f_3$} & 37    & 31    & 24    & 8 \\
    1000VV & \multicolumn{1}{c}{$f_4$} & 21    & 47    & 24    & 10 \\
    其他 & \multicolumn{1}{c}{$f_4$} & 49    & 67    & 24    & 10 \\
    \multicolumn{2}{c}{平均值(每月)} & 22.33 & 24.83 & -     & - \\
    \bottomrule
    \end{tabular}%
  \label{tab:2yearproduction}%
\end{table}%

\ksection{目标函数}
如引言所述,用制造期、完工时间和、滞后时间和等多目标加权求和可以更好地展现该问题,即:
\[Z=\alpha C_{\max}+\beta\sum_{j=1}^n C_{2,[j]}+\gamma\sum_{j=1}^n T_j
\]
式中的$\alpha,\beta,\gamma$位非负权重系数,可以用下面的法则确定:
\begin{compactenum}[(1)]
\item $C_{\max}$和$\sum C_{2,[j]}$不一定有相同的值域。为了减小差异,可以采用Framinan(2002)提出的调节程序。先假设一项有n个独立作业工作的期望完成时间为$C_{\max}/2$,因此$nC_{\max}\approx\sum C_{2,[j]}$,相当于将$C_{\max}$和$\sum C_{2,[j]}$与$n/2$和$1$分别相乘。
\item 实践中,权重系数的确定很大程度上取决于决策者(Taboada 和Coit,2008)。在案例的工厂中,完工时间和可以帮助评估及时交付情况,而滞后时间和主要作为超时加班计划的目的。一版安排完工时间和与滞后时间和为制造期的2倍。所以,$C_{\max},\sum C_{2,[j]}$和$\sum T_j$需分别再乘以1,2,2。如$n=12$,则权重为$(\alpha,\beta,\gamma)=(6,2,2)=(0.6,0.2,0.2)$。
\end{compactenum}
\ksection{现行调度方法}
现行的调度方法工作流程如\reff{fig:csf}所示。
\begin{figure}[h]
\centering
\includegraphics[width=16cm]{current_flow.jpg}
\caption{现行调度方法工作流程\label{fig:csf}}
\end{figure}
生产管理员确认了交付期后,销售人员接受顾客的订单并将订单通知发给生产管理员。当订单通知下达车间时,MRP 中心发布采购订单(PO)和制造单(MO)。通常工期以周计算,要比交付期$(D_j)$提前$5\ days\times 8\ h=40\ h$,即$d_j=D_j-40$。所有的采购订单需以PO 的基础立即呈给供应商,同时,生产调度需立即按MO 的基础制定。目前,作业的顺序按EDD 分派规则制定,并且单件流(Li 和Rong,2009)将在两个工段按相同顺序执行。

当换线时间等于0时,EDD 规则是最小化最大延迟的最优解(Baker 和Magazine,2000)。然而,现今的车间中,许多作业换线时间和产品簇换线时间会在单件流的情况下产生,所以EDD 规则通常会得到较大的完工时间。进一步说,在单件流情况下,4为工人同时在有限的空间内操作(见\reff{fig:cwf}),时常会相互干涉,产生可相当可观的闲置时间,使得处理时间从$p_a=24\ h/4=6\ h$变为$p_a=9\ h$(见\reft{tab:2yearproduction})。因此,现行调度方法将产生下列问题:
\begin{compactenum}[(1)]
\item 单件流会导致物料短缺,产生不能满足调度的处理困难;
\item 频繁换线和闲置使得流程时间变长和更多的超时加班;
\item 由于按期完成率低,不能处理紧急订单。
\end{compactenum}

为了解决这些问题,我们提出了作业划分和批量作业的的策略,将在接下来的章节里讨论。
% !Mode:: "TeX:UTF-8"
% !TEX root = ..\Literature_Translation.tex
\kchapter{建立解决方法}
本章我们将建立该问题的解决方法。首先,我们先将引入作业划分概念,并结合批量作业以减少换线时间、提高装配车间效率。而后,我们建立两个最优化特点,并提出了一个整数规划(MIP)公式以导出最优解。最后,我们提出了三个启发式方法以得到近似最优解。
\ksection{作业划分和批量作业}
为了改善现行的调度方法,我们引入了多批次发货(Bukchin,2002)概念,并打算用作业划分策略将作业划分成几个子作业。将作业划分策略结合入批量处理后,$c$项子作业便可分开同时由$c$个工人来处理。这样一来,可以避免相互干涉,减少等待时间。由于这些子作业都在批量中心,并且是连续离开机器,这$c$位工人必须同时停止作业,否则一些工人可能会处于闲置状态,而另一些工人却还在忙。
可以运用模块化设计和标准化以确保这$c$位工人同时停止作业,这样一来所工作拥有同样的处理时间,$c$位工人可以的工作强度近似一致。
如有必要,这4位工人可以互相帮工,这样一来,这些子作业可以同时完成。
因此,处理可以变得更为流畅,并且在工段1不会产生瓶颈。如\reff{fig:bwf}所示,接下来整个系统可以看作一个两阶段流水车间,工段1有4工人操作的批处理机器,工段2有2工人操作的离散机器。

承前所述,我们在工段1将作业划分成$c$项子作业,必须由$c$位工人同时操作,这样他们可以看作一个容量为$c$项作业的批量处理机器(Liu 和Yu,2000)。批$i$的处理时间为在此批中的最长作业处理时间,即$p_1(B_i)=\max p_{1j}=p_1$。
此外,每当一个新的批量产生,都需要考虑批量换线时间$s_b$,同样,开始首项作业或切换不同产品簇的作业时,需要考虑产品簇换线时间$s_{f1}$。
需要注意的是,批量中的最后一个作业或者同产品簇的下批首个作业不需要考虑产品簇换线时间。如此一来,批次$i$的完工时间$C_{1,[i]}$等于批量启动时间加上批量换线时间加上同批的产品簇换线和作业处理时间,即:
\begin{gather}
C_{1,[0]}=0 \label{equ:1} \\
C_{1,[i]}=C_{1,[i-1]} + s_b + s_{f1}\sum_{r=1}^c k_{1,[i,r]} + p_1 \label{equ:2}\\
(i=1,...,b, \text{如果}f_{1,[i,r]}=f_{1,[i,r-1]}\ k_{1,[i,r]}=0, \text{否则}k_{1,[i,r]}=1 )\notag
\end{gather}

作业在工段2用离散机器处理,作业按批进入但是一个接一个离开。开始首项工作或者切换产品簇的时候需要考虑产品簇换线时间$s_{j2}$。$J_{[j]}$的完工时间$C_{2,[j-1]}$等于$C_{1,[i]}$或者$C_{2,[j-1]}$加上$J_{[j]}$的产品簇换线时间和作业处理时间,即:
\begin{gather}
C_{2,[0]}=0\label{equ:3}\\
C_{2,[j]}=\max\{C_{1,[i]},C_{2,[j-1]}\} + s_{f2}\times k_{2,[j]} + p_{2,[j]}\label{equ:4}\\
(i=1,...,b,j=1,...,n \text{如果}f_{2,[j]}=f_{2,[j-1]}\ k_{2,[j]}=0, \text{否则}k_{2,[j]}=1)\notag 
\shortintertext{如此一来,}
C_{\max}=C_{2,[n]}\label{equ:5}\\
\sum_{j=1}^n T_j = \sum_{j=1}^n \max\{C_{2,[j]}-d_{[j]},0\}\label{equ:6}\\
Z = \alpha C_{\max} + \beta\sum_{j=1}^n C_{2,[j]} + \gamma\sum_{j=1}^n T_j  \label{equ:7}
\end{gather}

为了提高一次完成率,我们只接受滞后和小于$20\ h$每周的调度,因为滞后作业可以适当的周末里加班解决。否则,生产管理员需要和销售人员协商更改滞后作业的完工时间并重新调度。
\ksection{性质发掘}
\newcounter{prop}\newcounter{exam}
%\renewcommand{\theprop}{\arabic{prop}.}
\theoremheaderfont{\heiti}
\newtheorem{propetry}[prop]{性质}
\newtheorem{example}[exam]{例}

定义有$c$项作业的批为一个完整批,并称不足量的为局部批。回想一下,我们假设作业的数量为批量的倍数,即$n=b\times c$。

\begin{propetry}
对于案例问题来说,最优调度中的所有批皆为完整批。
\end{propetry}
\begin{proof}
记$S'$为一个包含一些完整批和局部批的序列,即$B_1,B_2,...,B'_k,B'_{k+1},...,B'_{b'}$。我们将第2个局部批$B'_{k+1}$的部分作业移到第1个局部批$B'_k$中,使之成为一个完整批$B_k$。重复这个步骤直到我们得到一个包含所有完整批($B_1,B_2,...,B_k,B_{k+1},...,B_b$)的序列$S$。这样一来,$S$中的作业开始时间要小于(对于$B'_k$之后的那些作业)等于(对于剩下的那些作业)$S'$,完工时间亦复如是。因此,就目标函数值来说,$S\text{主导了}S'$,证明完毕。
\end{proof}
\begin{propetry}
在最优调度中,同批中相同产品簇的作业必须连续处理。
\end{propetry}
\begin{proof}
读者可以参阅Baker(1999)和Chandru(1993)等的相关文献。
\end{proof}

在这两个性质的前提下,我们可以缩小MIP 模型和启发式方法的研究范围,以求得作业量为批量倍数的最优或近似最优调度,并将大大节省计算时间。
此外,我们采用$80/20$混合MTO/MTS 法则以应对紧急插单。
因此,当MTO 中的作业不足1批的时候,
生产管理员可以从MTS 中选取作业将其补足。

\ksection{~MIP 公式}
在MIP公式中的变量定义如下:
\begin{itemize}
\itemsep=0pt\parskip=0pt\parsep=0pt
\item 作业批次分派

$X_{j,[i,r]}$,0--1变量,在工段1如果$J_j$安排入批$i$的第$r$位置,那么取$1$,否则取$0$。
\item 簇换线时间

$k_{1,[i,r]}$,0--1变量,在工段1如果在批$i\text{第}r$位置的作业需要考虑簇换线时间,那么取$1\text{,否则取}0$。
\item 辅助变量

$g_{i,[r,]}$,0--1变量,用于产生$k_{1,[t,r]}$。
\end{itemize}

同时,完工时间和滞后时间皆为非负连续变量。这样一来,由\eqref{equ:1} -- (\ref{equ:7})即上述的两个性质,案例问题可以描述成如下MIP 模型:
\begin{gather}
\text{Minimize}\qquad \alpha C_{\max}+\beta\sum_{j=1}^n C_{2,[j]}+\gamma\sum_{j=1}^n T_j \label{equ:8}
\end{gather}
\begin{align}
&\text{s.t.}\notag\\
& \sum_{j=1}^n\sum_{r=1}^c X_{j,[i,r]} = c,\quad i=1,...,b\label{equ:9}\\
& \sum_{i=1}^b\sum_{r=1}^c X_{j,[i,r]} = 1,\quad j=1,...,n\label{equ:10}\\
& \sum_{j=1}^n X_{j,[i,r]} = 1,\quad i=1,...,b, r=1,...,c\label{equ:11}\\
&f_{1,[i,r]} = \sum_{j=1}^n X_{j,[i,r]}f_j,\quad i=1,...,b, r=1,...,c\label{equ:12}\\
&f_{2,[(i-1)\times c + r]} = \sum_{j=1}^n X_{j,[i,r]}f_j,\quad i=1,...,b, r=1,...,c\label{equ:13}\\
&d_{[(i-1)\times c + r]} = \sum_{j=1}^n X_{j,[i,r]}f_j,\quad i=1,...,b, r=1,...,c\label{equ:14}\\
&p_{2,[(i-1)\times c + r]} = \sum_{j=1}^n X_{j,[i,r]}p_{2j}\label{equ:15}\\
&k_{1,[1,1]} = 1\label{equ:16}\\
&f_{1,[i,r]}-f_{1,[i,r-1]} = - 3g_{i,[r,1]} - 2g_{i,[r,2]} - g_{i,[r,3]} + g_{i,[r,5]} + 2g_{i,[r,6]} + 3g_{i,[r,7]},\quad i=1,...,b, r=2,...,c\label{equ:17}\\
&g_{i,[r,1]} + g_{i,[r,2]} + g_{i,[r,3]} + g_{i,[r,4]} + g_{i,[r,5]} + g_{i,[r,6]} + g_{i,[r,7]} = 1,\quad i=1,...b, r=2,...,c\label{equ:18}\\
&k_{1,[i,r]} = g_{i,[r,1]} + g_{i,[r,2]} + g_{i,[r,3]} + g_{i,[r,5]} + g_{i,[r,6]} + g_{i,[r,7]},\quad i=1,...b, r=2,...,c\label{equ:19}\\
&f_{1,[i,1]}-f_{1,[i-1,c]} = - 3g_{i,[1,1]} - 2g_{i,[1,2]} - g_{i,[1,3]} + g_{i,[1,5]} + 2g_{i,[1,6]} + 3g_{i,[1,7]},\quad i=2,...,b\label{equ:20}\\
&g_{i,[1,1]} + g_{i,[1,2]} + g_{i,[1,3]} + g_{i,[1,4]} + g_{i,[1,5]} + g_{i,[1,6]} + g_{i,[1,7]} = 1,\quad i=2,...b\label{equ:21}\\
&k_{1,[i,1]} = g_{i,[1,1]} + g_{i,[1,2]} + g_{i,[1,3]} + g_{i,[1,5]} + g_{i,[1,6]} + g_{i,[1,7]},\quad i=2,...b\label{equ:22}\\
&k_{2,[(i-1)\times c + r]} = k_{1,[i,r]},\quad i=1,...,b, r=1,...,c\label{equ:23}\\
&C_{1,[0]} = 0\label{equ:24}\\
&C_{1,[i]} = C_{1,[i-1]} + s_b + s_{f1}\sum_{r=1}^c k_{1,[i,r]} + p_1,\quad i=1,...,b\label{equ:25}\\
&C_{2,[0]} = 0\label{equ:26}\\
&C_{2,[(i-1)\times c + r]} = \max \{C_{1,[i]},C_{2,[(i-1)\times c + r]}\} + s_{f2}k_{2,[(i-1)\times c + r]} + p_{2,[(i-1)\times c + r]},\quad i=1,...b-1, r=1,...,c\label{equ:27}\\
&C_{\max} = C_{2,[n]}\label{equ:28}\\
&T_j = \max\{C_{2,[j]}-d_{[j]},0\},\quad j=1,...,n\label{equ:29}\\
&X_{j,[i,r]},k_{1,[i,r]},k_{2,[j]} = 0 \text{或} 1,\quad j=1,...,n, i=1,...,b, r=2,...,c\label{equ:30}\\
&g_{i[r,l]} = 0 \text{或} 1,\quad i=1,...,b, r=2,...,c, l=1,...,7\label{equ:31}
\end{align}

考虑到\eqref{equ:27}和\eqref{equ:29}是非线性的,但他们可以较为容易的转化为线性形式。例如,约束条件(\ref{equ:27})可以改写为
\begin{numcases}{}
C_{2,[(i-1)\times c + r]}\geqslant C_{1,[i]} + s_{f2}k_{2,[(i-1)\times c + r]} + p_{2,[(i-1)\times c + r]} \notag\\
C_{2,[(i-1)\times c + r]}\geqslant C_{2,[(i-1)\times c + r - 1]} + s_{f2}k_{2,[(i-1)\times c + r]} + p_{2,[(i-1)\times c + r]}\notag
\end{numcases}

我们现在来解释一下这个MIP 公式。目标函数(\ref{equ:8})是最小化加权制造期和、完工时间和、总滞后时间和三者之和。约束条件(\ref{equ:9})确保每一批次正好有$c$项作业,约束(\ref{equ:10})和(\ref{equ:11})确保作业和批次的位置一一对应。\eqref{equ:12}确定了工段1中批$i$第$r$位置的作业簇,约束(\ref{equ:13}) -- (\ref{equ:15})确定了安排在工段2的作业产品簇、工期和处理时间,\eqref{equ:16} -- (\ref{equ:22})确定了$k_{1,[i,r]}$。
由于\eqref{equ:17}左边为非0--1变量,即$f_{1,[i,r]}-f_{1,[i,r-1]}=\{-3,-2,-1,0,1,2,3\}$,需要增加辅助变量$g_{i,[r,l]}$和约束(\ref{equ:17}) -- (\ref{equ:22})以确保该模型为一个MIP 问题(Hillier 和Liberman,2001)。
如果在工段1作业$[i,r]$和其前继作业属于同一簇,那么$k_{1,[i,r]}=0\text{,否则}k_{1,[i,r]}=1$。约束条件(\ref{equ:23})产品簇换线变量$k_{2,[j]}$。
由于所有作业流程在系统中的顺序一致,所以$k_{2,[j]}=k_{1,[i,r]}$。\eqref{equ:24}和(\ref{equ:25})确定了工段1的完工时间,\eqref{equ:26}和(\ref{equ:27})确定了工段2的完工时间,并确保作业只有在其前继作业和本身完成了工段1的处理后才能开始操作。\eqref{equ:28}和(\ref{equ:29})确定了案例问题的运行情况度量,\eqref{equ:30}和(\ref{equ:31})是0--1变量约束。

\begin{example}
作为一个例证,考虑一个有12项工作的问题,$f_j=(3,2,3,4,1,4,2,2,3,3,4,4),d_j=(100,160,160,100,\\
180,140,140,140,140,140,180,140),s_b=6.4,s_{f1}=3.2,s_{f2}=1.6$,其他数据见\reft{tab:2yearproduction}。\label{exam:1}
\end{example}

可以用LINGO 10 来解决这个MIP 问题,我们可以得到最优结果如下:

在$(\alpha,\beta,\gamma) = (0.6,0.2,0.2)$下的目标函数值为
$$Z = 331.04, C_{\max} = 164.0, {\textstyle\sum_j C_j = 1163.2}, T_{\max} = 0$$

各指派值为
\begin{align*}
 X_{10,[1,1]} & = 1 & X_{3,[1,2]} & = 1 & X_{1,[1,3]} & = 1 & X_{9,[1,4]} & = 1\\
 X_{4,[2,1]} & = 1 & X_{11,[2,2]} & = 1 & X_{12,[2,3]} & = 1 & X_{6,[2,4]} & = 1\\
 X_{8,[3,1]} & = 1 & X_{7,[3,2]} & = 1 & X_{2,[3,3]} & = 1 & X_{5,[3,4]} & = 1
\end{align*}

所有其他的指派值为0。

产品簇换线变量为
\begin{align*}
k_{1,[1,1]} & = 1 & k_{1,[2,1]} & = 1 & k_{1,[3,1]} & = 1 & k_{1,[3,4]} & = 1 \\
k_{2,[1]} & = 1 & k_{2,[5]} & = 1 & k_{2,[9]} & = 1 & k_{2,[12]} & = 1
\end{align*}

所有其他的产品簇换线变量皆为0。最优调度的Gantt 图如\reff{fig:gantt1}所示。
\begin{figure}[h]
\includegraphics[width=10cm]{ganttexam1.jpg}
\caption{例\ref{exam:1}最优调度的Gantt 图\label{fig:gantt1}}
\end{figure}

这个MIP 公式可以用来求解只有少量的作业和产品簇,对于通常问题的调度求解十分困难。实际情况中,我们可能遇到大量作业和产品簇的问题,所以建立启发式方法变得十分重要,这些方法可以较快解决中、大型问题的近似最优解。

\ksection{启发式方法}
在2.5节中,我们解释了现行的单件处理和EDD 规则调度方法。为了有效解决涌现的问题和提高调度效果,我们提出了启发式方法。
\newcommand{\Step}{{\heiti 步骤}}

\ksubsection{启发式方法1:FBEDD}
我们将EDD 规则结合完整批的性质,建立了一个完整批EDD (FBEDD)启发式方法,以最小化目标值,具体步骤如下

\begin{asparaenum}
\renewcommand{\labelenumi}{\heiti 步骤\theenumi~}
\item 置$S=\phi,\ S_U = \{J_1,J_2,...,J_n\},\ B=\phi,\ i=1,2,...,b,\ b = n/c$。
\item 将$S_U$中的作业按$d_j$非减的顺序排列,置$i=1$。
\item 将$S_U$中的前$c$项作业组成批次$B_i$,即$B_i = \{J_{[1]},...,J_{[c]}\}$。
\item 如果$S_U = \phi$,则执行\Step5,否则$i=i+1$,并执行\Step 3。
\item 最终的调度为$S=B=(B_1,...,B_b)$。
\end{asparaenum}

\ksubsection{启发式方法2:FBFS}
作业要群组成一些产品簇的集合。我们选用同簇中的$c$项作业组成一个完整批,然后把不同簇中剩余的作业组成最后一批,这批的作业按最短处理时间(SPT)规则排序。这个启发式方法的步骤叫做完整批簇排序(FBFS)启发式算法,如下所示。其中$\# F^f$表示产品簇作业$f$的集合$F^f$的作业数量。
\begin{asparaenum}
\renewcommand{\labelenumi}{\heiti 步骤\theenumi~}
\item 置$S_1 = \phi,\ S_2=\phi,\ S_U = \{J_1,...,J_n\},\ B = \phi,\ B' = \phi,\ b=n/c$。
\item 将$S_U$中的作业按$d_j$非减的顺序排列。
\item 将得到的序列作业按产品簇排序以组成集合,置$f=1$。
\item 如果$\# F^f \geqslant c$,执行\Step 7,否则执行\Step 5。
\item 置$i=1$,将$F^f\text{并入}B'_i$作为一个局部批,然后从$S_U\text{剔除}B'_i$。
\item 如果$S_U = \phi$,执行\Step 9,否则置$f=f+1$,并执行\Step 4。
\item 置$i=1$,从$F^f$中拿出前$c$项作业作为一个完整批$B^f_i$,将至并入$B$,并从$S_U\text{中剔除}B^f_i$。
\item 如果$\# F^f \geqslant c$,执行\Step 7,否则执行\Step 5。
\item 置$B = \{B_1^1,...,B_i^1,...,B_1^4,...,B_i^4\},\ B' = \{B'_1,...,B'_4\}$。按$p_{2j}$非减的顺序排列重排$B'$中的作业。
\item 在作业顺序不变的情况下,重新定$B\text{与}B'$中的指数。置$B=\{B_1,...,B_{k-1}\}\text{并且}B' = \{B_k,B_{k+1},...,B_b\}$均为完整批。
\item 最终调度为$S=B\cup B' = \{B_1,B_2,..,B_b\} = \{J_{[1]},J_{[2]},...,J_{[n]}\}$。
\end{asparaenum}

我们来解释以上的步骤。在\Step1和\Step2中,我们构建了一个初始调度,作业是按EDD 规则排序的。在\Step3中,作业群组成4个簇。\Step4 -- 8,我们取各簇前$c$项作业组成一个完整批,并将剩余的作业组成一个最后批。在\Step9,我们按SPT 规则重排了局部批的作业,以改良目标函数值。\Step10我们在不改变作业顺序的情况下,重新设置了各完整批的指数,以保证其完整。在\Step11,我们得到了最终调度方案。
\begin{example}
考虑一个12个作业的问题,其中$p_{aj}=9,s_a=5.2,p_{1j}=24,s_b=6.4,s_{f1}=3.2,s_{f2}=1.6$。其他数据同例\ref{exam:1}。
\end{example}
\begin{asparaenum}
\renewcommand{\labelenumi}{解\theenumi:}
\item 用EDD 规则求解该问题,得到最终调度:$C_{\max}=211,\ \sum C_j = 1419.0,\ \sum T_j = 66.2,\ Z(S)=408.32$。
\item 用FBEDD 规则求解该问题,得到最终调度:$C_{\max}=178.4,\ \sum C_j = 1344.4,\ \sum T_j = 0,\ Z(S)=375.92$。
\item 用FBFS 规则求解该问题,得到最终调度:$C_{\max}=164,\ \sum C_j = 1163.2,\ \sum T_j = 0,\ Z(S)=331.04$。
\end{asparaenum}
\ksubsection{启发式方法3:RFBFS}
当我们运用FBFS 求解大型问题的时候,换线时间和可以被最小化,然而,这个方法给出了同簇产品的优先级,却不顾工期。因此,这可能会耗费许多时间处理同簇产品的批次,导致一些作业被提前完成,而一些作业却要滞后完成。为了改善FBFS 的这个不足,我们融入了滚动时域调度策略以达成滚动完整批产品簇调度(RFBFS)的启发式方法。RFBFS 将大型问题分割成多个EDD 序列的小型问题,然后重复执行FBFS 直到所有作业都被调度。
由于RFBFS 运行FBFS 处理小型短时域问题,这样可以用于处理大型问题。该启发式方法的步骤如下,其中$n_b$表示每月可处理的批数。
\begin{asparaenum}
\renewcommand{\labelenumi}{\heiti 步骤\theenumi~}
\item 置$S_1 = \phi,\ S_U = \{J_1,...,J_n\},\ B_1 = \phi,\ B'_1 = \phi,\  n = b \times c$。
\item 将$S_U$中的作业按$d_j$非减的顺序排列,并置$\nu = 1$。
\item 从$S_U$中拿出前$n_b/2$项作业来执行FBFS,并产生一个局部调度$S_1 = B_1 \cup B'_1$。然后,置$\nu = \nu + 1$,并把剩下的$n_b/2$项作业执行FBFS,产生下一个局部调度$S_2 = B_2 \cup B'_2$。
\item 如果$S_U=\phi$,执行\Step5,否则,置$\nu = \nu + 1$,执行\Step3。
\item 最终的调度为$S=S_1\cup S_2\cup...\cup S_{\nu}$。
\end{asparaenum}

\ksection{确定下界}
为了评估启发式方法得到结果的质量,本节将引出下界。由于这些都是多目标函数,所以下界包含3个部分:\begin{inparaenum}[(1)]
\item $C_{\max}\text{的下界}LB_1$;
\item $\sum C_j\text{的下界}LB_2$;
\item $\sum T_j\text{的下界}LB_3$。
\end{inparaenum}
\renewcommand{\labelenumi}{}
\begin{asparaenum}
\item $C_{\max}\text{的下界}(LB_1)$
\suspend{asparaenum}

Ahmadi(1992)提出了LPT-Johnson 法则以最小化处理多簇产品的离散批系统的制造期。根据这个法则,作业在工段2按处理时间排列为非增序列。
为了得到合理的下界,我们融入了Azizoglu 和Webster(2003)提出的获得下界进程。
我们考虑将批换线时间$s_b$和产品簇换线时间$s_{f1}$计入工段1的处理时间,放宽工段2产品簇换线时间。
这样一来,下界可以由制造期加最小工段2簇换线时间($f_n \times s_{f2}$)得到。
记$C_{2,[j]}$(LPT)为$H_{[j]}$在工段2的完工时间,$f_n$为所有作业簇的数量,即:
\begin{numcases}{}
C_{2,[1]}(LPT) = (s_b + s_{f1} + p_1) + p_{2,[1]}\notag\\
C_{2,[j]}(LPT) = C_{2,[j-1]} + p_{2,[j]},\quad j=2,...,n\notag\\
LB_1 = C_{2,[n]}(LPT) + f_ns_{f2}\notag
\end{numcases}
\resume{asparaenum}
\item $\sum C_j\text{的下界}(LB_2)$
\suspend{asparaenum}

Ahmadi(1992)同时也提出完整批最短处理时间(Full Batch-SPT)法则以最小化工段2无闲置的离散批系统完工时间和。
根据这个法则,作业在工段2按处理时间排列为非减序列。
记$C_{2,[j]}$(SPT)为作业在工段2的完工时间,这样一来下界可由完工时间和加上工段2最小簇换线时间得到。簇换线时间只有在首个作业和末尾作业$f_n-1$时才考虑,即:
\begin{numcases}{}
C_{2,[1]}(SPT) = s_b + s_{f1} + p_1 + p_{2,[1]}\notag\\
C_{2,[j]}(SPT) = C_{2,[j-1]} + p_{2,[j]},\quad j=2,...,(n-f_n+1)\notag\\
C_{2,[j]}(SPT) = C_{2,[j-1]} + s_{f2} + p_{2,[j]},\quad j=(n-f_n+2),...,n\notag\\
LB_2 = \sum_{j=1}^n C_{2,[j]}(SPT) + (n+\sum_{k=1}^{f_n}(f_n-k))s_{f2}\notag
\end{numcases}
\resume{asparaenum}
\item $\sum T_j\text{的下界}(LB_3)$
\end{asparaenum}

Baker 和Martin(1974)证明SPT序列最小化了当所有作业都滞后的单机器滞后时间和。在这个离散批系统中,批处理时间相同,这样考虑由此引起的滞后和时,这个系统可以考虑为单个机器。
记$C_{2,[j]}$(SPT)为完工时间,$d_{[j]}$为SPT 顺序下$J_{[j]}$的工期。
这样一来,滞后时间和的下界可以由放宽所有作业皆滞后的约束得到,即:
\begin{numcases}{}
T_j = \max \{C_{2,[j]}(SPT)-d_{[j]}(SPT),0\},\quad j=1,...,n \notag \\
LB_3 = \sum_{j=1}^n T_j\notag
\end{numcases}

将这3个独立的下界综合,可以得到该问题的下界:
\[ LB = \alpha LB_1 + \beta LB_2 + \gamma LB_3
\]

为了评估启发式算法求解的质量,我们用相对误差(RER)定义启发式方法和下界的差别:
\[
RER = \left( \frac{Heuristic_i - LB}{LB}\right)\times 100\%
\]
其中$Heuristic_i$是各启发式方法的目标函数值。
\begin{example}
举一个下界的例子,考虑一个12个作业的问题,其中$p_{1j}=24,s_b=6.4,s_{f1}=3.2,s_{f2}=1.6$,其他数据同例\ref{exam:1}。
\end{example}

在参数为$(\alpha,\beta,\gamma)=(n/2,2,2)$的情况下,得到的下界为:
\begin{align*}
LB &= 0.6 \times LB_1 + 0.2 \times LB_2 + 0.2 \times LB_3 \\
& =0.6 \times 164 + 0.2 \times 1152 + 0.2 \times 0\\
& = 328.8 
\end{align*}
3种启发式方法的相对误差如下:
\begin{align*}
RER_{EDD} & = \left(\frac{408.32-328.8}{328.8}\right)\times 100 \% = 24.18 \%\\
RER_{FBEDD} & = \left(\frac{375.92-328.8}{328.8}\right)\times 100 \% = 14.33 \%\\
RER_{FBFS} & = \left(\frac{311.04-328.8}{328.8}\right)\times 100 \% = 24.18 \%
\end{align*}

从结果来看,$RER_{FBFS}$的值较小,说明由FBFS 得到的结果质量明显要优于EDD 和FBEDD。
% !Mode:: "TeX:UTF-8"
% !TEX root = ..\Literature_Translation.tex
\kchapter{计算实验}
在本章,将通过计算实验来评估MIP 模型和所提出的启发式算法的运行情况。
参考实际数据\reft{tab:2yearproduction},装配车间每月大约能装配24件作业,故实验可以通过3部分进行:\begin{inparaenum}[(1)]
\item 时间域为$2$周的包含$n=12$件作业的小型问题;
\item 时间域为$1\text{或}2.5$月的包含对应$n=24\text{或}n=60$件作业的中型问题;
\item 时间域为$4,8,12$月的包含对应$n=100,200,300$件作业的中型问题。
\end{inparaenum}
所有启发式方法皆用MATLAB 语言编写程序,并在处理器为AMD Athlon 2.91 GHz 的PC 上运行。
\ksection{评估小型问题}
我们随机产生10个12件作业和4个产品簇的示例($N=10$),其中$f_j$由均匀分布$DU(1,4)$产生,$d_j$由均匀分布$DU(100,180)$产生。
为了评估这些启发式方法的效果,考虑两个度量方面:求解质量和运行时间(Hamta,2012)。
为了检测启发式方法的求解质量,我们可以计算平均百分相对偏差(ARPD)如下(Laha 和Sarin,2009):
\[ARPD = \frac{100}{N}\sum_{i=1}^N\left(\frac{Heuristic_i - Optimal_i}{Optimal_i}\right)
\]
其中,$Heuristic_i$是第$i$个示例用启发式方法得到的目标函数值,$Optimal_i$是其MIP 目标函数值。
我们计算3个启发式方法EDD,FBEDD 和FBFS 按$(\alpha,\beta,\gamma)=(n/2,2,2)=(12/2,2,2)=(0.6,0.2,0.2)$加权的目标函数值和ARPD 的结果如\reft{tab:comparisonforarpd}所示。
\begin{table}[h]
  \centering\xiaowu
  \caption{MIP 和3个启发式方法的ARPD 比较}
    \begin{tabular}{ccccccccc}
    \toprule
    \multirow{2}[4]{*}{示例编号} & \multicolumn{2}{c}{MIP} & \multicolumn{2}{c}{EDD} & \multicolumn{2}{c}{FBEDD} & \multicolumn{2}{c}{FBFS} \\
   \cline{2-9}
          & Z     & time(s) & Z     & ARPD  & Z     & ARPD  & Z     & ARPD \\
      \midrule
    1     & 331.04 & 10    & 423.64 & 27.97 & 375.92 & 13.56 & 331.04 & 0 \\
    2     & 374.56 & 15    & 435.88 & 16.37 & 396.88 & 5.96  & 376.48 & 0.51 \\
    3     & 338.72 & 21    & 425.64 & 25.66 & 380.08 & 12.21 & 344.48 & 1.7 \\
    4     & 332.4 & 123   & 399.48 & 20.18 & 354.32 & 6.59  & 338.16 & 1.73 \\
    5     & 377.28 & 35    & 435.84 & 15.52 & 395.2 & 4.75  & 377.28 & 0 \\
    6     & 342.64 & 47    & 441.2 & 28.76 & 393.68 & 14.9  & 342.64 & 0 \\
    7     & 316.64 & 24    & 414.64 & 30.95 & 349.64 & 10.42 & 316.64 & 0 \\
    8     & 345.36 & 22    & 443.28 & 28.35 & 391.48 & 13.35 & 345.36 & 0 \\
    9     & 359.04 & 35    & 451.04 & 25.62 & 399.44 & 11.25 & 359.04 & 0 \\
    10    & 386.72 & 8     & 442.56 & 14.44 & 432.8 & 11.92 & 389.04 & 0.6 \\[3pt]
    平均值   & 350.44 & 34    & 431.32 & 23.38 & 386.94 & 10.49 & 352.02 & 0.45 \\
    \bottomrule
    \end{tabular}
  \label{tab:comparisonforarpd}
\end{table}
由最后一行各ARPD 可以看出,EDD 为$23.38\%$,在区间$(14.44,30.95)\%$中,FBEDD 为$10.49\%$,在区间$(4.75,14.90)\%$中,而FBFS 仅为$0.45\%$,在区间$(0,1.73)\%$中,较其他两者小,因此,FBFS 的运行效果要明显优于其他两个。
至于运行时间,MIP 每次需要$34\ s$,在区间$(8,123)\ s$中,而FBEDD 和FBFS 皆只需小于$0.3\ s$。综上所述,FBFS 是针对小型问题的好方法。
\ksection{评估中、大型问题}
考虑紧急插单的情况,设计成由$80\%$的MTO(面向订单生产)作业和$20\%$的MTS(面向库存生产)作业组成的中、大型问题。我们将最短交付期$D_j$设置成至少15天($15\times 8 = 120\ h$),最长交付期设置成1年($365\times 8 = 3000\ h$),所以,交付期期间为$[120,3000]\ h$。
工期$d_j$的确定基于交付期和能力限制,正常工期($d_j = D_j$)设置成基于普通能力,紧张工期($d_j = D_j - 40$)和松弛工期($d_j = D_j$)设置成基于稀缺和宽裕能力。

由于这个问题是NP 困难,我们不能在可接受的时间内找到中、大型问题的最优解。
为了评估启发式方法的求解质量,我们用相对误差(RER)来比较启发式方法和下界的差。进一步说,我们定义相对改善率(RIR)来比较启发式方法和现行方法的运行差异(Mokhtari,2010)。
\[RIR(\%) = \frac{100}{N}\sum_{i=1}^N \left(\frac{Heuristic_i - Best_i}{Best_i}\right)
\]
其中,$Heuristic_i$是用启发式方法的第$i$个示例的求解值,$Best_i$是示例中的最优值。

对于中、大型问题,需要测试15类问题、5类作业($n = 24,60,100,200,300$)和3个水准的工期(紧张(T),正常(N),松弛(L))的组合。
每一个问题都包含10个示例,其中$f_j$由$DU(1,4)$产生,对于5类作业,$D_j$分别由$DU(120,360),DU(120,270),DU(120,1120),DU(120,2120),DU(120,3000)$产生。
我们用3种启发式方法计算$n=24,d_j=D_j$,及权重为$(\alpha,\beta,\gamma)=(n/2,2,2)=(24/2,2,2)=(0.75,0.125,0.125)$的问题的下界和目标值,汇总RER 和RIR 见\reft{tab:comparen=24}。
\begin{table}[h]
  \centering\xiaowu
  \caption{通过10个示例比较$n=24, d_j = D_j$问题的三种启发式方法}
    \begin{tabular}{cccccccccc}
    \toprule
    \multirow{2}[4]{*}{示例编号} & 下界(LB)    & \multicolumn{3}{c}{目标值(Z)} & \multicolumn{3}{c}{RER (\%)} & \multicolumn{2}{c}{RIR (\%)} \\
    \cline{2-10}
          &       & (1) EDD & (2) FBEDD & (3) RFBFS &$\frac{(1) - LB}{LB}$ & $\frac{(2) - LB}{LB}$ & $\frac{(3) - LB}{LB}$ &$\frac{(1) - (3)}{(3)}$ & $\frac{(2) - (3)}{(3)}$ \\
          \midrule 
    1     & 661.5 & 953.4 & 748.9 & 698.9 & 44.13 & 13.22 & 5.66  & 36.41 & 7.15 \\
    2     & 723.3 & 1047.6 & 824.2 & 744.4 & 44.84 & 13.95 & 2.92  & 40.73 & 10.72 \\
    3     & 723.2 & 998.2 & 825.3 & 765.8 & 38.02 & 14.11 & 5.88  & 30.35 & 7.77 \\
    4     & 722.7 & 956.5 & 811.6 & 743.5 & 32.36 & 12.31 & 2.88  & 28.66 & 9.17 \\
    5     & 762.2 & 1029.9 & 898.9 & 808.5 & 35.12 & 17.94 & 6.07  & 27.38 & 11.18 \\
    6     & 661.2 & 918.9 & 733.4 & 683.9 & 38.98 & 10.92 & 3.43  & 34.37 & 7.24 \\
    7     & 761.1 & 963.3 & 846.8 & 791.3 & 26.58 & 11.26 & 3.97  & 21.74 & 7.01 \\
    8     & 732.8 & 997.7 & 850.3 & 790.6 & 36.15 & 16.03 & 7.89  & 26.19 & 7.54 \\
    9     & 748.4 & 934   & 830.9 & 791.3 & 24.8  & 11.03 & 5.73  & 18.03 & 5.01 \\
    10    & 663.9 & 934.8 & 742   & 709.9 & 40.8  & 11.76 & 6.92  & 31.68 & 4.53 \\[3pt]
    平均值   & 716   & 973.4 & 811.2 & 752.8 & 35.95 & 13.3  & 5.14  & 29.31 & 7.76 \\
    \bottomrule
    \end{tabular}
  \label{tab:comparen=24}
\end{table}
由最后一行可以看出,EDD 和FBEDD 分别有平均RER $35.95\%\text{和}13.30\%$,而RFBFS 为$5.14\%$。同时可以看出,RFBFS 比EDD 和FBEDD 好,因为两者的平均RIR 分别为$29.31\%\text{和}7.76\%$。因此,RFBFS 就求解质量将提高效果上来说要优于其余两者。


\kchapter{总结和展望}
