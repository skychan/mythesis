% !Mode:: "TeX:UTF-8"
% !TEX root = ..\Literature_Translation.tex
%%  绪论
\chapter{绪论}
\section{课题研究的背景和意义}
本文的研究对象是汽车零部件装配生产线,是典型的多品种小批量生产方式,并且在需求日益多样化的背景下,时常要根据产品更换生产线。本文将从这种生产方式着手研究多品种产品的装配生产调度问题。

\subsection{课题研究背景}
人类随着生产 力的发展不断探寻着新的生产制造模
式。2O 世纪 20年代,在管理大师泰勒“科学管理”思想的影
响下,原本以手工制造为核心的制造模式向 “少品种大批
量”转变。美国的亨利·福特通过弓l进“流水线”概念,为社会
带来了大批量生产模式最直接的益处,即高效生产下的大
量产品以及单件产品低廉的战本。之后.随着人们对产品的
需求从原始拥有向多元化丰富化转变,原本的 “单件大批
量”⋯已不冉能满足人~f3x+产品的需求。为适应这种转变,多
品种小批量贝4成为必然趋势 .也只有通过这种柔性化生产
模式才能降低成本、减少库存,从而更好地适应市场变化。

课题针对某公司装配车间进行调度研究,涉及精益生产、生产计划与控制、基础工业工程、成本控制等内容,与工业工程专业紧密联系;课题面向具体的企业问题,能够解决具体的工程实际问题,能够培养和锻炼学生运用所学理论工具和创新思维解决实际工程问题的能力,具有较强的工程意义。

生产调度就是组织执行生产进度计划的工作。在过去的几十年中,人们对调度问题进行了大量的研究工作。从上个世纪50年代起,调度问题的研究就受到应用数学、运筹学、工程技术等领域科学家的重视,科学家们利用运筹学中的线性规划、整数规划、目标规划、动态规划及决策分析方法,研究并
解决了一系列有代表意义的调度和优化问题 8 但
是,人们普遍把 -6CZEX ,
.E@Z977 和 .G779U 三人有
关调度的研究工作
[ " ] 作为调度理论研究的正式开
始,他们 > 人也被人们称为调度理论的奠基人 8 此\cite{徐俊刚2004生产调度理论和方法研究综述}
后\cite{1998distributed}

我刚刚开始写不久...
我打算先讲一下多产品装配 然后将一下调度的发展 再具体讲已有的算法
董老师 14:43:18 
差不多
sky 14:44:25 
翻译写完了,感觉课题研究的范围有点大 所以背景和意义不太好写
sky 14:59:22 
老师 我写背景可以从哪些方面入手
sky 14:59:43 
感觉没有一个实例很难说的清楚
董老师 15:02:44 
研究背景和意义是比较难写的
你面向的对象是汽车零部件装配线  然后你的调度是有几个目标的 可以从对象和目标入手来写
同时也可以参考一些相关的硕士论文和博士论文 看他们的研究背景意义怎么写的

\subsection{课题研究意义}
调度是一种决策形式,在制造业扮演者至关重要的角色,尤其在现今充满竞争的环境下,有效的生产调度已成为能在市场竞争下生存的必须。


企业面临着交货期提前、客户需求多样化和产品生命周期缩短的
压力,因此缩短交货期、提高资源利用率、降低生产成本、提高生产
运作的灵活性,已成为保证企业市场竞争力的重要手段。本文以日本
大金现有装配车间为实际应用背景,对该装配车间存在的问题进行有
针对性的研究。通过解决这些问题来提高生产线效率,缩短生产周期,
提高市场反应能力。
\section{课题研究的国内外现状}
\subsection{国内研究状况}

\subsection{国外研究状况}
装配车间的生产调度决定了产品能否准时交货,某企业装配车间有多条装配线,每条装配线可分时轮番生产不同品种的产品,为了管理方便,按照产品所属厂家来安排组织生产,但这种方法存在负荷不均衡、换线损失等问题。课题对现在存在的问题进行分析,对所有需要生产的产品进行筛选组合,设计新的调度方案,以期获得均衡的生产线利用率、减少换线时间浪费、缩短完工时间、降低生产成本