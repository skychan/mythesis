% !Mode:: "TeX:UTF-8"
% !TEX root = ..\thesis.tex
\chapter{总结和展望}

\section{总结}
本文的前几章由某汽车电子有限公司的装配车间现存问题,提出改进方案,并将之抽象成多品种多装配线轮番调度优化模型,然后设计了多个算法对该模型进行求解,并通过Python 脚本进行了计算实验,评价了求解结果及算法的适用性。

由第二章的装配车间生产现状及其问题分析,所提出的打破主机厂限制的解决方
案,引导出了多品种多装配线的混流生产调度问题,并在第三章进行相关模型建立,其中模型2 要比模型1 多考虑插单的情况,第五章的实验从一些方面说明了这两个模型设计的合理性。第四章从初始解的生成和解的改进设计了Cyc 类的算法,根据流水线之间和流水线内部调整的策略不同,可以得到不同的求解算法。值得一提的是,这一章中本文提出了虚拟序列的概念,并将之运用到了算法的设计中,并且由实验表明,使用了虚拟序列策略的算法Vtr -- Tabu 和VVT 算法都能得到质量较高的调度结果。
\section{展望}
多品种装配车间调度问题有许多方面是本文没有涉及的,仅从改进方案的方面来看,本文仍可以在换线轮番装配策略基础上,考虑各流水线的混流生产情况,进一步均衡生产。
由于本文在建模的时候假设了诸多限制,例如订单的不可中断假设,逐步消除这些假设的限制,可以使模型更为接近实际情况。
此外,在算法设计上,调度规则的制定是一个较为复杂的过程,而且还有许多其他领域的算法可以用于车间调度问题,需要去探索。