% !Mode:: "TeX:UTF-8"
% !TEX root = ..\thesis.tex
\chapter{计算实验及算法评估}
本章将对上一章的算法进行评估,通过设计相关计算实验,建立评价体系,具体评估各算法的效果及其适用规模。然后根据实验结果,为研究对象制定合适的调度方案。
\section{算法评价体系}

到达时间:Poisson分布
处理时间:负指数分布(指数分布)、艾尔朗分布

由\reft{tab:2jobshopinfo},流水线的数量可以是$5,6,7$,产品品种的范围是$29\~ 777$,所以合理的作业数量为$30,50,100,200,500,1000$
,作业处理时间单位为转换过的 $t.u.$,由于每个订单的作业批量都比较大,所以订单处理时间单位为$500 t.u.$
产品特点是体积小,而每个订单所含的产品数量,即需要考虑的作业数量大,一般在$1000\~ 2000$左右
\section{实验设计}
\lstset{	basicstyle = \tiny\ttfamily,
	keywordstyle = \color{blue}\bfseries,
	stringstyle = \color{red},
	emph = {solve},
	emphstyle = \color{Green}\bfseries,
	commentstyle = \color{CadetBlue}
	}
\begin{lstlisting}[language = Python]
import sys
sys.path.append(".\\functions")
import generate
from collections import namedtuple
Item = namedtuple("Item", ['process','due','wt','wc'])

def  h(lambda1,lambda2,tardiness,completion,wt,wc):			# define the contribution of one item for the obj function
	value = lambda1*wt*tardiness + lambda2*wc*completion
	return value

def solve(input_data):
	Data = input_data.split('\n')					# load data
	n = len(Data) -1						# get the amount of items
	print n
	items = []
	for j in xrange(n):
		data = Data[j]
		parts = data.split()
		p = int(parts[0])					# get the process time
		s = int(parts[2])						# get the setup time
		d = int(parts[3])					# get the due date
		wt = int(parts[4])					# get the tardiness weights
		wc = int(parts[5])					# get the completion weights
		items.append(Item(p+s,d,wt,wc))			# combine those item data
	print 'Data loaded!'
	J = range(n)
	m = 5
	S = []
	a = []
	tl = []
	L = []
	c = [None]*n
	for l in xrange(m):
		S.append([])
		a.append(0)
		tl.append(0)
	t = 0
	f = open(".\\result\\sky",'w')
	while J:
		if 0 in a:
			l_star = a.index(0)
			p,d,wt = [],[],[]
			for j in J:				
				item = items[j]
				p.append(item.process)
				d.append(item.due)
				wt.append(item.wt)
			orderidx = generate.Idx(t,p,d,wt)
			j_star = J[orderidx.index(max(orderidx))]
			S[l_star].append(j_star)
			J.remove(j_star)
			L.append(j_star)
			tl[l_star] = t + items[j_star].process
			c[j_star] = tl[l_star]
			a[l_star] = 1
		else:
			t_star = min(tl)
			for l in xrange(m):
				if tl[l] == t_star:
					a[l] = 0
			t = t_star
	print 'initial sloution done!'
	f.write(str(S) + '\n' + str(L) + '\n' + str(c))
	f.close()


import sys
if __name__ == '__main__':
	if len(sys.argv) > 1:
		file_location = sys.argv[1].strip()
#		output = sys.argv[2].strip()
		input_data_file = open(file_location, 'r')
		input_data = ''.join(input_data_file.readlines())
		input_data_file.close()
		solve(input_data)
\end{lstlisting}

\section{结果及评估}


\section{小结}